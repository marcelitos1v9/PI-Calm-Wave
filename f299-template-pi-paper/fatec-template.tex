%%%% fatec-article.tex, 2024/03/10

%% Classe de documento
\documentclass[
  a4paper,%% Tamanho de papel: a4paper, letterpaper (^), etc.
  12pt,%% Tamanho de fonte: 10pt (^), 11pt, 12pt, etc.
  english,%% Idioma secundário (penúltimo) (>)
  brazilian,%% Idioma primário (último) (>)
]{article}

%% Pacotes utilizados
\usepackage[]{fatec-article}
\Author{1}{Name={Bruno Lopes de Souza\\João Pedro Faustino Cordeiro\\ Leonardo Wicher L. Ferreira\\ Marcelo Augusto Aguiar da Cruz}}

\Author{2}{Name={\{ bruno.souza204@fatec.sp.gov.br \}\\ \{ joao.cordeiro2@fatec.sp.gov.br \} \\ \{ leonardo.ferreira39@fatec.sp.gov.br\} \\ \{ marcelo.cruz21@fatec.sp.gov.br \}}}

%% Definição das palavras-chaves/keywords
\Keyword{1}{Artigo}{Article}
\Keyword{2}{Latex}{Latex}
\Keyword{3}{Informática}{Informatic}

%%%% Resumo no idioma primário (brazilian)
\begin{Abstract}[brazilian]%% Idioma (brazilian ou english)
  O resumo FATEC Registro é uma descrição completa e concisa dos componentes-chave da metodologia do estudo e dos achados importantes da pesquisa. Normalmente, o resumo é o primeiro encontro do leitor com uma pesquisa ou relato, sendo algumas vezes o único elemento recuperado e/ou revisado nas bases de dados científicos. Esse elemento provê a primeira impressão, muitas vezes a mais importante, identificando o valor potencial ou a relevância do enfoque da pesquisa e dos resultados. Se o resumo for bem escrito, ele atrairá leitores para obter uma cópia do manuscrito completo que será incorporado aos que já foram encontrados, e seu trabalho será citado. Se o resumo for mal escrito, a pesquisa poderá ser ignorada ou, até mesmo, esquecida.
\end{Abstract}

%%%% Resumo no idioma secundário (english)
\begin{Abstract}[english]%% Idioma (brazilian ou english)
The summary is a complete and concise description of the key components of the study methodology and important research findings. Typically, the summary is the reader's first encounter with a research report or paper, and sometimes it is the only element retrieved and/or reviewed in scientific databases. This element provides the first impression, often the most important one, identifying the potential value or relevance of the research approach and findings. If the summary is well-written, it will attract readers to obtain a copy of the full manuscript that will be incorporated into those already found, and your work will be cited. If the summary is poorly written, the research may be ignored or even forgotten.
\end{Abstract}

%% Processamento de entradas (itens) do índice remissivo (makeindex)
\makeindex%

%% Arquivo(s) de referências
\addbibresource{fatec-article.bib}

%% Início do documento
\begin{document}

% Seções e subseções
%\section{Título de Seção Primária}%

%\subsection{Título de Seção Secundária}%

%\subsubsection{Título de Seção Terciária}%

%\paragraph{Título de seção quaternária}%

%\subparagraph{Título de seção quinária}%

\section*{Introdução}%
\label{sect:intro}
Os Objetivos de Desenvolvimento Sustentável (ODS), estabelecidos pela Organização das Nações Unidas (ONU),
têm como objetivo promover o desenvolvimento sustentável em suas três dimensões: social, econômica e ambiental.
Dentre as 17 metas estabelecidas, a meta três foca na promoção da saúde e do bem-estar.

O Processamento Auditivo Central (PAC) refere-se à capacidade do cérebro de interpretar e dar sentido aos sons que chegam através do ouvido,
desempenhando um papel crucial na comunicação e na aprendizagem. Essa habilidade auditiva é fundamental para a compreensão da fala,
a localização de sons e a discriminação auditiva em ambientes com ruídos. No entanto, quando essa capacidade é comprometida,
surge o Transtorno do Processamento Auditivo Central (TPAC), uma condição que afeta a forma como o cérebro processa as informações sonoras \cite{Silva2023}.
Além disso, problemas de comunicação podem surgir, dificultando a interação social \cite{palfery2007}. Assim,
o TPAC não apenas impacta o desempenho acadêmico, mas também pode levar ao isolamento social.

Diante desse cenário, o presente trabalho propõe o desenvolvimento de um dispositivo antirruído com emissão de frequências calmantes,
projetado para reduzir ruídos externos e proporcionar um ambiente sonoro mais controlado.
Essa abordagem visa melhorar a qualidade de vida de crianças com TPAC, visto que ambientes sonoros adequados podem facilitar a compreensão auditiva
e diminuir a sobrecarga sensorial, promovendo assim uma melhor aprendizado.

\section*{OBJETIVO} \label{sect:obj}

Nosso estudo visa desenvolver uma tecnologia para reduzir o estresse sonoro em ambientes escolares,  
com foco em crianças com TPAC.  
Pretende-se criar um fone de ouvido adaptativo que se ajuste dinamicamente para mitigar o impacto dos sons externos,  
proporcionando às crianças um ambiente propício ao aprendizado. Reconhecendo a dificuldade de crianças com TPAC  
devido à exposição a estímulos sonoros, visamos em nosso estudo compreender os desafios específicos enfrentados  
por essas crianças nesse contexto, identificando as fontes de ruído e os momentos do dia em que o estresse sonoro é mais pronunciado.

O fone de ouvido adaptativo que será desenvolvido será uma solução para esses desafios.  
Ele será capaz de receber o som ambiente e aplicar técnicas de equalização para ajustar dinamicamente suas características,  
proporcionando um ambiente auditivo mais confortável e tranquilo para as crianças. Além disso, o fone de ouvido poderá emitir  
diferentes tipos de ruídos, como ruído marrom, branco ou rosa, dependendo da situação, ajudando a mascarar sons indesejados  
e criar um ambiente mais relaxante. O controle sobre as configurações do fone de ouvido estará disponível através de um aplicativo móvel,  
permitindo ajustes personalizados conforme necessário e upload de novos ruídos ou sons personalizáveis ao usuário.

Além do desenvolvimento do fone, pretende-se realizar uma avaliação de sua eficácia e relevância em ambientes escolares reais.  
Isso incluirá testes em salas de aula, onde será coletado o feedback de professores, terapeutas e pais,  
além das próprias crianças com TPAC. O principal desafio e objetivo é entender como essa tecnologia pode ser integrada  
de maneira eficaz na rotina escolar para promover um ambiente mais inclusivo e favorável ao aprendizado.

No final, espera-se que nossa tecnologia melhore a qualidade de vida e o bem-estar emocional das crianças com TPAC,  
além de contribuir para a promoção da educação inclusiva, facilitando a participação e o sucesso acadêmico de todas as crianças,  
independentemente de suas necessidades sensoriais ou cognitivas.

\section*{Objetivos Específicos}

\begin{enumerate}
    \item \textbf{Identificar as Fontes de Ruído}: Compreender quais são os principais tipos de sons que causam estresse e distração em crianças com TPAC em ambientes escolares.
    
    \item \textbf{Desenvolver Prototótipos do Fone de Ouvido}: Criar protótipos do fone de ouvido e relizar testes.
    
    \item \textbf{Testar Eficácia em Ambientes Reais}: Conduzir testes de campo em salas de aula para avaliar como o fone de ouvido se comporta em situações reais.
    
    \item \textbf{Analisar Impacto no Aprendizado}: Avaliar como a utilização do fone de ouvido adaptativo pode melhorar a capacidade de aprendizado das crianças com TPAC.
\end{enumerate}

\section*{ESTADO DA ARTE} \label{sect:estadoarte}

\input{Topicos/estadodaarte}

\section*{METODOLOGIA} \label{sect:metodologia}

\input{Topicos/metodologia}

\section*{RESULTADOS PRELIMINARES}\label{sect:resultados}

\input{Topicos/resultados}

\section*{CONCLUSÃO}\label{sect:conclusao}

\input{Topicos/conclusao}

\printbibliography

%% Elementos pós-textuais (opcionais): Apêndice e Anexo
%Caso for utilizar, basta retirar o símbolo de % na frente do comando
%\input{./Extras/post-textual}

%% Fim do documento
\end{document}