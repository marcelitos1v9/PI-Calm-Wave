Nosso estudo visa desenvolver uma tecnologia para reduzir o estresse sonoro em ambientes escolares,  
com foco em crianças com TPAC.  
Pretende-se criar um fone de ouvido adaptativo que se ajuste dinamicamente para mitigar o impacto dos sons externos,  
proporcionando às crianças um ambiente propício ao aprendizado. Reconhecendo a dificuldade de crianças com TPAC  
devido à exposição a estímulos sonoros, visamos em nosso estudo compreender os desafios específicos enfrentados  
por essas crianças nesse contexto, identificando as fontes de ruído e os momentos do dia em que o estresse sonoro é mais pronunciado.

O fone de ouvido adaptativo que será desenvolvido será uma solução para esses desafios.  
Ele será capaz de receber o som ambiente e aplicar técnicas de equalização para ajustar dinamicamente suas características,  
proporcionando um ambiente auditivo mais confortável e tranquilo para as crianças. Além disso, o fone de ouvido poderá emitir  
diferentes tipos de ruídos, como ruído marrom, branco ou rosa, dependendo da situação, ajudando a mascarar sons indesejados  
e criar um ambiente mais relaxante. O controle sobre as configurações do fone de ouvido estará disponível através de um aplicativo móvel,  
permitindo ajustes personalizados conforme necessário e upload de novos ruídos ou sons personalizáveis ao usuário.

Além do desenvolvimento do fone, pretende-se realizar uma avaliação de sua eficácia e relevância em ambientes escolares reais.  
Isso incluirá testes em salas de aula, onde será coletado o feedback de professores, terapeutas e pais,  
além das próprias crianças com TPAC. O principal desafio e objetivo é entender como essa tecnologia pode ser integrada  
de maneira eficaz na rotina escolar para promover um ambiente mais inclusivo e favorável ao aprendizado.

No final, espera-se que nossa tecnologia melhore a qualidade de vida e o bem-estar emocional das crianças com TPAC,  
além de contribuir para a promoção da educação inclusiva, facilitando a participação e o sucesso acadêmico de todas as crianças,  
independentemente de suas necessidades sensoriais ou cognitivas.

\section*{Objetivos Específicos}

\begin{enumerate}
    \item \textbf{Identificar as Fontes de Ruído}: Compreender quais são os principais tipos de sons que causam estresse e distração em crianças com TPAC em ambientes escolares.
    
    \item \textbf{Desenvolver Prototótipos do Fone de Ouvido}: Criar protótipos do fone de ouvido e relizar testes.
    
    \item \textbf{Testar Eficácia em Ambientes Reais}: Conduzir testes de campo em salas de aula para avaliar como o fone de ouvido se comporta em situações reais.
    
    \item \textbf{Analisar Impacto no Aprendizado}: Avaliar como a utilização do fone de ouvido adaptativo pode melhorar a capacidade de aprendizado das crianças com TPAC.
\end{enumerate}