Os Objetivos de Desenvolvimento Sustentável (ODS), estabelecidos pela Organização das Nações Unidas (ONU),
têm como objetivo promover o desenvolvimento sustentável em suas três dimensões: social, econômica e ambiental.
Dentre as 17 metas estabelecidas, a meta três foca na promoção da saúde e do bem-estar.

O Processamento Auditivo Central (PAC) refere-se à capacidade do cérebro de interpretar e dar sentido aos sons que chegam através do ouvido,
desempenhando um papel crucial na comunicação e na aprendizagem. Essa habilidade auditiva é fundamental para a compreensão da fala,
a localização de sons e a discriminação auditiva em ambientes com ruídos. No entanto, quando essa capacidade é comprometida,
surge o Transtorno do Processamento Auditivo Central (TPAC), uma condição que afeta a forma como o cérebro processa as informações sonoras \cite{Silva2023}.
Além disso, problemas de comunicação podem surgir, dificultando a interação social \cite{palfery2007}. Assim,
o TPAC não apenas impacta o desempenho acadêmico, mas também pode levar ao isolamento social.

Diante desse cenário, o presente trabalho propõe o desenvolvimento de um dispositivo antirruído com emissão de frequências calmantes,
projetado para reduzir ruídos externos e proporcionar um ambiente sonoro mais controlado.
Essa abordagem visa melhorar a qualidade de vida de crianças com TPAC, visto que ambientes sonoros adequados podem facilitar a compreensão auditiva
e diminuir a sobrecarga sensorial, promovendo assim uma melhor aprendizado.